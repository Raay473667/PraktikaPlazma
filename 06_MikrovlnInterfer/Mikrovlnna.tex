\documentclass[a4paper,12pt]{article}
\usepackage [utf8x]{inputenc}
\usepackage[czech]{babel}
\usepackage{graphicx}
\usepackage{amsmath}
\usepackage{siunitx}
\usepackage{xspace}
\usepackage{url}
\usepackage{indentfirst}
\usepackage[margin=22mm]{geometry}
\usepackage{esvect}
\usepackage{ragged2e}
\usepackage{tikz,pgf}
\usepackage{bm}
\usepackage{perpage}
\usepackage{capt-of}

\graphicspath{
	{img/}
	{plots/}
}

\newcommand{\e}{\text{e}}


\MakeSorted{figure}
\newtoks\jmenopraktika \newtoks\jmeno \newtoks\datum
\newtoks\obor \newtoks\skupina \newtoks\rocnik \newtoks\semestr
\newtoks\cisloulohy \newtoks\jmenoulohy
\newtoks\tlak \newtoks\teplota \newtoks\vlhkost
\jmenopraktika={Mikrovlnná interferometrie plazmatu}  % nahradte jmenem vaseho predmetu
\jmeno={Radek Horňák, Lukáš Vrána}            
\datum={5. 4. 2022}        % nahradte datem mereni ulohy                           
\rocnik={2.}                  
\semestr={IV.}                 
\cisloulohy={6}    % cislo ulohy           

\begin{document}
	\begin{center}
		{\Large Přírodovědecká fakulta Masarykovy univerzity} \\
		\bigskip
		{\Large \bfseries PRAKTIKUM Z~FYZIKY PLAZMATU} \\
		\bigskip
		{\Large \the\jmenopraktika}
	\end{center}
	\bigskip
	\noindent
	\setlength{\arrayrulewidth}{1pt}
	\begin{tabular*}{\textwidth}{@{\extracolsep{\fill}} l l}
		\large {\bfseries Zpracovali:}  \the\jmeno  \hspace{20mm} \large  
		{\bfseries Naměřeno:} \the\datum\\[2.5mm]
		\hline
	\end{tabular*}

\section{Teorie}
Plazma lze obecně kvalitativně považovat za vodič, dielektrikum či magnetickou kapalinu. Výběr modelu je závislý na konkrétní situaci. V případě interakce elektromagnetického záření s plazmatem se v oblasti nízkých frekvencí plazma popisuje jako vodič pomocí nízkofrekvenční vodivost, při vysokých frekvencích je vhodná aplikace dielektrického modelu včetně definice vysokofrekvenční permitivity. Hranicí mezi nízkými a vysokými frekvencemi je plazmová frekvence $\omega_\text{pl}$, od které se může vlna plazmatem šířit. Ta souvisí s hustotou plazmatu pomocí vztahu

\begin{equation}
	\omega_\text{pl} = \frac{n_\text{e} e^2}{m_\text{e} \epsilon_0}
\end{equation}
kde $n_\text{e}$ je hustota volných elektronů, $e$ elementární náboj, $m_\text{e}$ hmotnost elektronu a $\epsilon_0$ permitivita vakua.

Pro dielektrický model nemagnetického plazmatu je permitivita komplexní skalár. V případě, že pro popis rozdělení rychlosti elektronů zvolíme Maxwellovo rozdělení, je relativní permitivita popsaná vztahem

\begin{equation}
	\epsilon_\text{r} = 1- \frac{n_\text{e} e^2 (\omega - i \nu_\text{m})}{m_\text{e} \epsilon_0 \omega (\omega^2 +\nu_\text{m}^2)}
	\label{komplexnipermitivita}
\end{equation}
kde $\nu_\text{m}$ je srážková frekvenci pro přenos hybnosti elektron--neutrál. Na rozdíl od běžných dielektrik je reálná část permitivity plazmatu menší než jedna. Místo relativní permitivity můžeme obdobně popisovat plazma pomocí indexu lomu $n$. Jeho reálná část je přímo úměrný fázové rychlosti vlny a tedy i fázovému posuvu $\Delta\phi$. Platí vztah

\begin{equation}
 	\Delta\phi = k_0(1-n) \Delta z
\end{equation}
kde k$_0$ je vlnové číslo a $\Delta z$ kus dráhy.  

\subsection{Stanovení hustoty elektronů}
Pro stanovení hustoty elektronů aproximujeme vztah pro relativní permitivitu \eqref{komplexnipermitivita} tak, že zanedbáme imaginární složku a vypustíme $\nu_\text{m}^2)$. Dostáváme

\begin{equation}
	\epsilon_\text{r} = 1- \frac{n_\text{e} e^2 (\omega - i \nu_\text{m})}{m_\text{e} \epsilon_0 \omega (\omega^2 +\nu_\text{m}^2)}
	\label{permitivita}
\end{equation}



%za $\Delta z$ rozměr co zabírá výbojka ve vlnovodu
\section{Měření a výsledky}


\section{Závěr}


\end{document}
