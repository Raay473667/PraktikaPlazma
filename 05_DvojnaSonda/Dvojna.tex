\documentclass[a4paper,12pt]{article}
\usepackage [utf8x]{inputenc}
\usepackage[czech]{babel}
\usepackage{graphicx}
\usepackage{amsmath}
\usepackage{siunitx}
\usepackage{xspace}
\usepackage{url}
\usepackage{indentfirst}
\usepackage[margin=22mm]{geometry}
\usepackage{esvect}
\usepackage{ragged2e}
\usepackage{tikz,pgf}
\usepackage{bm}
\usepackage{perpage}
\usepackage{capt-of}

\graphicspath{
	{img/}
	{plots/}
}

\newcommand{\e}{\text{e}}


\MakeSorted{figure}
\newtoks\jmenopraktika \newtoks\jmeno \newtoks\datum
\newtoks\obor \newtoks\skupina \newtoks\rocnik \newtoks\semestr
\newtoks\cisloulohy \newtoks\jmenoulohy
\newtoks\tlak \newtoks\teplota \newtoks\vlhkost
\jmenopraktika={Studium kladného sloupce doutnavého
	výboje pomocí elektrostatických sond: dvojná sonda}  % nahradte jmenem vaseho predmetu
\jmeno={Radek Horňák, Lukáš Vrána}            
\datum={5. 4. 2022}        % nahradte datem mereni ulohy                           
\rocnik={2.}                  
\semestr={IV.}                 
\cisloulohy={6}    % cislo ulohy           

\begin{document}
	\begin{center}
		{\Large Přírodovědecká fakulta Masarykovy univerzity} \\
		\bigskip
		{\Large \bfseries PRAKTIKUM Z~FYZIKY PLAZMATU} \\
		\bigskip
		{\Large \the\jmenopraktika}
	\end{center}
	\bigskip
	\noindent
	\setlength{\arrayrulewidth}{1pt}
	\begin{tabular*}{\textwidth}{@{\extracolsep{\fill}} l l}
		\large {\bfseries Zpracovali:}  \the\jmeno  \hspace{20mm} \large  
		{\bfseries Naměřeno:} \the\datum\\[2.5mm]
		\hline
	\end{tabular*}

\section{Teorie}
\subsection{Dvojná sonda}
Jednou z možných konstrukcí Langmuirovy sondy je takzvaná dvojná
sonda. Ta se může skládat například ze dvou jednoduchých válcových
sond stejných rozměrů. Mezi nimi by měla být dostatečná vzdálenost, 
aby se nepřekrývaly jejich stěnové vrstvy a také by měly být ve
stejných regionech plazmatu. V porovnání s jednoduchou sondou její
VA charakteristiky vždy vykazuje strmou oblast v okolí nulového
napětí. Zároveň saturovaný iontový proud limituje proud obvodem,
sonda tak méně narušuje samotné plazma.

V našem případě měříme pomocí dvojné symetrické sondy, obě její
části jsou umístěné v ekvipotenciální ploše plazmatu. Schématické
znázornění sondy je na obr. \ref{schemadvojna}. Sonda se ustavuje
na plovoucím potenciálu $V_\text{{fl}}$. Měříme cirkulační proud
$i_\text{{d}}$ okruhem sond při přiloženém napětí $V_\text{{d}}$
mezi ně. 

VA charakteristika ideální dvojné rovinné sondy je na
obr. \ref{charakteristikadvojna}. V bodě $A$, kde platí 
$V_\text{{d}} = 0$ a $i_\text{{d}} = 0$, se obě sondy nachází na
témže plovoucím potenciálu $V_\text{{fl}}$. 

$V_\text{{d}} < 0 $ kolem bodu $B$ je
oblast takzvaného záporného napětí. Platí zde
\begin{equation}
	\sum i_\text{{p}} + \sum i_\text{{e}} = 0
\end{equation}
Potenciál první ze sond se blíží potenciálu plazmatu,
potenciál druhé sondy bude nižší než plovoucí.

Kolem bodu $C$ platí $V_\text{{d}} \ll 0 $, jedná se tedy
o oblast VA charakteristiky kde je velké záporné napětí.
První sonda přebírá veškerý tok elektronů, druhá sonda
je silně negativní vzhledem k potenciálu plazmatu.
Pokud dále vzrůstá $V_\text{{d}}$, dojde k nasycení
iontového proudu druhé sondy a celkový proud vnějším
okruhem $i_\text{{d}}$ zůstává konstantní.

\subsection{Měření a výsledky}


\ref{VAmerenidvojna}


\section{Měření a výsledky}


\section{Závěr}


\end{document}
