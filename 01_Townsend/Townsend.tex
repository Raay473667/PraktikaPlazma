\documentclass[a4paper,12pt]{article}
\usepackage [utf8x]{inputenc}
\usepackage[czech]{babel}
\usepackage{graphicx}
\usepackage{amsmath}
\usepackage{siunitx}
\usepackage{xspace}
\usepackage{url}
\usepackage{indentfirst}
\usepackage[margin=22mm]{geometry}
\usepackage{esvect}
\usepackage{ragged2e}
\usepackage{tikz,pgf}
\usepackage{bm}
\usepackage{perpage}
\usepackage{capt-of}

\graphicspath{
	{img/}
	{plots/}
}

\MakeSorted{figure}
\newtoks\jmenopraktika \newtoks\jmeno \newtoks\datum
\newtoks\obor \newtoks\skupina \newtoks\rocnik \newtoks\semestr
\newtoks\cisloulohy \newtoks\jmenoulohy
\newtoks\tlak \newtoks\teplota \newtoks\vlhkost
\jmenopraktika={Měření prvního Townsendova koeficientu}  % nahradte jmenem vaseho predmetu
\jmeno={Radek Horňák, Lukáš Vrána}            
\datum={1. 3. 2022}        % nahradte datem mereni ulohy                           
\rocnik={2.}                  
\semestr={IV.}                 
\cisloulohy={6}    % cislo ulohy           

\begin{document}
	\begin{center}
		{\Large Přírodovědecká fakulta Masarykovy univerzity} \\
		\bigskip
		{\Large \bfseries PRAKTIKUM Z~FYZIKY PLAZMATU} \\
		\bigskip
		{\Large \the\jmenopraktika}
	\end{center}
	\bigskip
	\noindent
	\setlength{\arrayrulewidth}{1pt}
	\begin{tabular*}{\textwidth}{@{\extracolsep{\fill}} l l}
		\large {\bfseries Zpracovali:}  \the\jmeno  \hspace{20mm} \large  
		{\bfseries Naměřeno:} \the\datum\\[2.5mm]
		\hline
	\end{tabular*}

\section{Teorie}

Teorie lavin popsaná Townsendem vysvětluje základní ionizační mechanismus elektrického výboje. Mějme dvě paralelní kovové desky a mezi nimi homogenní elektrické pole $E$. Elektrony jsou v~poli urychlovány a sráží se s~neutrálními částicemi, přičemž může docházet k~nepružným srážkám vedoucím k~excitaci nebo ionizaci neutrálů. Pokud počet elektronů v~místě $x$ označíme $n$, pak podél dráhy d$x$ vznikne ionizačními srážkami d$n$ nových elektronů a platí

\begin{equation}
	dn = n \alpha dx
	\label{1}
\end{equation}

kde $\alpha$ je označení pro první Townsendův, někdy nazývaný i ionizační koeficient. Ten vyjadřuje počet ionizačních srážek jednoho elektronu na jednotkové délce. Integrací získáme

\begin{equation}
	\ln n = \alpha x + konst.
	\label{2}
\end{equation}

a následnou úpravou dostáváme vztah

\begin{equation}
	n = n_0 e^{\alpha x}
	\label{3}
\end{equation}

kde $n_0$ je počet elektronů v~počátečním bodě $x$ = 0. Ionizační koeficient závisí na intenzitě elektrického pole $E$ a na tlaku plynu v~aparatuře $p$. Je li $E/p$ dáno, můžeme psát $\alpha$

\begin{equation}
	\alpha = p f \left( \frac {E}{p} \right) 
	\label{4}
\end{equation}

tedy ionizační koeficient je úměrný počtu srážek na jednotku délky. Experimentální výsledky ukazují, že konkrétní závislost $\alpha$ na $E/p$ je ve tvaru

\begin{equation}
	\frac{\alpha}{p} = A~e^{-\frac{Bp}{E}} 
	\label{5}
\end{equation}

kde $A$ a $B$ jsou konstanty, pro které platí

\begin{equation}
	U_i = \frac{B}{A}
	\label{6}
\end{equation}

kde $U_i$ je ionizační potenciál plynu v~aparatuře. Hodnotu konstant A~a B lze určit experimentálně.

\section{Měření a výsledky}

Aparatura použitá v~tomto praktiku schematicky znázorněná na obr. \ref{aparatura}. Její hlavní komponenty jsou zdroj napětí, rotační olejová vývěva, výbojka, jehlový ventil, Piraniho manometr, ampérmetr a voltmetr. Je založená na principu fotoelektrického jevu, pomocí rtuťové výbojky osvětlujeme hliníkovou rovinnou katodu UV zářením a produkujeme tak fotoelektrony. Ty jsou urychlovány homogenním elektrickým polem na mřížkovou anodu. Katodu můžeme posouvat a tím měnit dráhu, po níž dochází k~ionizaci neutrálů. Do výbojky, kterou čerpáme vakuovou vývěvou, je vpuštěný argon. Tlak se nastavuje jehlovým ventilem a měří Piraniho manometrem. Jedná se o~nepřímý manometr, pro argon je tedy tlak odečítaný z~manometru potřebné vynásobit faktorem 1,59.

\begin{figure}[h]
	\centering
	\includegraphics[width=130mm]{aparatura.png}
	\caption{Schéma použité aparatury}
	\label{aparatura}
\end{figure}

Při měření musíme dbát na to, aby ve výbojce nevznikl samostatný výboj, tedy měříme pro hodnoty intenzity elektrického pole 80-120~V/cm. To v~praxi udržujeme nastavením napětí na zdroji a přizpůsobením vzájemné vzdálenost katody a UV výbojky. Výstupem z~měření je poloha katody $x$, hodnota napětí $U$ a proud $i$ pro několik hodnot konstantní intenzity elektrického pole. Pro každou změnu intenzity pole naladíme irisovou clonu UV výbojky tak, abychom měli maximální proud okolo 1800 pA z~důvodu rozsahu na přístroji do 1999 pA. Proud je tedy řádově pA až nA, pro zlepšení přesnosti měření z~ampérmetru odečítáme vždy 3 hodnoty a dále budeme pracovat s~průměrem z~nich. Tlak je konstantní o~hodnotě $p$ = 79,5 Pa.

V~rovnicích (\ref{2}) a (\ref{3}) lze nahradit počet elektronů proudem. Z~naměřených dat můžeme sestavit graf závislosti $i = i_0 f(x)$, viz obr. \ref{ifx}. Body jsou proložené exponenciální funkcí $i = i_0 e^{\alpha x}$, z~toho získané $i_0$ a $\alpha$ jsou uvedené v~levé části tabulky \ref{tab1}.

Dále můžeme vytvořit graf závislosti ln$i$ = ln$i_0$ + $\alpha x$, viz obr. \ref{lni}. Závislost je proložená lineární funkcí $i$ = $i_0$ + $\alpha x$, získané $i_0$ a $\alpha$ jsou uvedené v~pravé části tabulky \ref{tab1}.

V~obou případech je vidět, že s~rostoucí intenzitou elektrického pole $E$ proud $i_0$ klesá a ionizační koeficient $\alpha$ roste.

\begin{figure}[h!]
	\centering
	\includegraphics[width=145mm]{ifx.png}
	\caption{Graf závislosti $i$ na $x$.}
	\label{ifx}
\end{figure}

\begin{figure}[h!]
	\centering
	\includegraphics[width=145mm]{lni.png}
	\caption{Graf závislosti ln$i$ na $x$.}
	\label{lni}
\end{figure}


\begin{center}
	\begin{table}[h]
		\centering
		\caption{Hodnoty proudů $i_0$ a ionizačních koeficientů $\alpha$ pro různé hodnoty $E$.}
		\label{tab1}
		\begin{tabular}{|c|c|c|c|c|} \hline
			\multicolumn{1}{|c|}{}  & \multicolumn{2}{c|}{$i = i_0 e^{\alpha x}$}& \multicolumn{2}{c|}{$i$ = $i_0$ + $\alpha x$}  \\ \hline
			$E$ [Vcm$^{-1}$] & $i_0$ [pA] & $\alpha$ [cm$^{-1}$] & $i_0$ [pA] & $\alpha$ [cm$^{-1}$] \\ \hline
			80 & 63,65 $\pm$ 6,62 & 1,60 $\pm$ 0,06 & 44,65 $\pm$ 1,06 & 1,83 $\pm$ 0,04\\ \hline
			90 & 45,30 $\pm$ 2,79 & 1,85 $\pm$ 0,03& 30,15 $\pm$ 1,11 & 2,12 $\pm$ 0,08\\ \hline
			100 & 26,65 $\pm$ 1,70 & 2,07 $\pm$ 0,03 & 15,32 $\pm$ 1,12 & 2,43 $\pm$ 0,09\\ \hline
			110 & 19,92 $\pm$ 1,23 & 2,23 $\pm$ 0,03 & 12,25 $\pm$ 1,12 & 2,60 $\pm$ 0,09 \\ \hline
			120 & 7,09 $\pm$ 0,48 & 2,80 $\pm$ 0,03 & 3,73 $\pm$ 1,28 & 3,23 $\pm$ 0,20 \\ \hline
			
		\end{tabular}
	\end{table}
\end{center}

\newpage
Jelikož jsme provedli měření pro několik hodnot $E/p$, můžeme sestavit grafy závislosti ln$\alpha/p$ = $f(p/E)$ a proložit jej lineární funkcí ln$\alpha/p$ = ln$A$ - $\frac{Bp}{E}$, která vychází z~úpravy rovnice (\ref{5}). To provedeme jak pro $\alpha$ získané z~exponenciálního fitu v~grafu \ref{ifx}, viz obr. \ref{exp}, tak i pro $\alpha$ z~lineárního fitu v~grafu \ref{lni}, viz obr.\ref{lin}. Následně pomocí rovnice (\ref{6}) určíme ionizační potenciál argonu $U_i$, viz tab. \ref{tab2}. Tabulková hodnota pro argon je $U_i$ = 15,76 eV. Té se více přiblížil potenciál $U_i$ = (14,50 $\pm$  2,38) eV získaný po dosazení $\alpha$ z~lineárního fitu.
 
 \begin{figure}[h]
 	\centering
 	\includegraphics[width=145mm]{exp.png}
 	\caption{Graf závislosti ln$\alpha/p$ na $p/E$ pro $\alpha$ z~exponenciálního fitu.}
 	\label{exp}
 \end{figure}

 \begin{figure}[h]
	\centering
	\includegraphics[width=145mm]{lin.png}
	\caption{Graf závislosti ln$\alpha/p$ na $p/E$ pro $\alpha$ z~lineárního fitu.}
	\label{lin}
\end{figure}
 
\begin{center}
	\begin{table}[h]
		\centering
		\caption{Hodnoty konstant $A$, $B$ a ionizačních potenciálů $U_i$ argonu z~ln$\alpha/p$ = ln$A$ - $\frac{Bp}{E}$.}
		\label{tab2}
		\begin{tabular}{|c|c|c|c|c|c|} \hline
			\multicolumn{3}{|c|}{Dosazení $\alpha$ z~exponenciálního fitu} & \multicolumn{3}{c|}{Dosazení $\alpha$ z~lineárního fitu}  \\ \hline
			$A$ & $B$ & $U_i$ [eV] & $A$ & $B$ & $U_i$ [eV] \\ \hline
			9,31 $\pm$ 1,16 & 156,40 $\pm$ 18,48 &  16,80 $\pm$ 2,89  & 10,99 $\pm$ 1,18 & 159,31 $\pm$ 19,80 & 14,50 $\pm$ 2,38\\ \hline

			
		\end{tabular}
	\end{table}
\end{center}

\newpage
\section{Závěr}


\end{document}
