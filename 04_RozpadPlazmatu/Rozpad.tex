\documentclass[a4paper,12pt]{article}
\usepackage [utf8x]{inputenc}
\usepackage[czech]{babel}
\usepackage{graphicx}
\usepackage{amsmath}
\usepackage{siunitx}
\usepackage{xspace}
\usepackage{url}
\usepackage{indentfirst}
\usepackage[margin=22mm]{geometry}
\usepackage{esvect}
\usepackage{ragged2e}
\usepackage{tikz,pgf}
\usepackage{bm}
\usepackage{perpage}
\usepackage{capt-of}
\usepackage{subcaption}

\graphicspath{
	{img/}
	{plots/}
}

\newcommand{\e}{\text{e}}


\MakeSorted{figure}
\newtoks\jmenopraktika \newtoks\jmeno \newtoks\datum
\newtoks\obor \newtoks\skupina \newtoks\rocnik \newtoks\semestr
\newtoks\cisloulohy \newtoks\jmenoulohy
\newtoks\tlak \newtoks\teplota \newtoks\vlhkost
\jmenopraktika={Studium rozpadu plazmatu mikrovlnnou metodou}  % nahradte jmenem vaseho predmetu
\jmeno={Radek Horňák, Lukáš Vrána}            
\datum={15. 3. 2022}        % nahradte datem mereni ulohy                           
\rocnik={2.}                  
\semestr={IV.}                 
\cisloulohy={6}    % cislo ulohy           

\begin{document}
	\begin{center}
		{\Large Přírodovědecká fakulta Masarykovy univerzity} \\
		\bigskip
		{\Large \bfseries PRAKTIKUM Z~FYZIKY PLAZMATU} \\
		\bigskip
		{\Large \the\jmenopraktika}
	\end{center}
	\bigskip
	\noindent
	\setlength{\arrayrulewidth}{1pt}
	\begin{tabular*}{\textwidth}{@{\extracolsep{\fill}} l l}
		\large {\bfseries Zpracovali:}  \the\jmeno  \hspace{20mm} \large  
		{\bfseries Naměřeno:} \the\datum\\[2.5mm]
		\hline
	\end{tabular*}

\section{Teorie}
\subsection{Difuze a rekombinace v plazmatu}
Důležitou charakteristikou plazmatu jakožto ionizovaného plynu je koncentrace elektronů a iontů. Pokud přestaneme dodávat energii, plazma se začne rozpadat, což se projeví postupným poklesem koncentrace nabitých částic. Tento pokles je způsoben buď difúzí a ná\-sled\-nou rekombinací na stěnách nebo objemovou rekombinací.

Řešením rovnice kontinuity pro koncentraci elektronů 

\begin{equation}
	\frac{\partial n}{\partial t} + \text{div} \overrightarrow{\Phi} = 0
\end{equation}
za předpokladu, že máme výbojku válcového tvaru s délkou větší než poloměrem, do\-stáváme koncentraci elektronů v jedné dimenzi jako

\begin{equation}
	n(x,t) = n_0(x)\,\e^{\left(-\frac{Dt}{\Lambda^2} \right)} 
\end{equation}
kde $n_0$ je koncentrace elektronů v počátku $x = 0$, $D$ je difúzní koeficient, $t$ je čas a $\Lambda$ je difuzní délka.
Radiální profil koncentrace je v tomto případě  

\begin{equation}
	n_0(x) = \text{konst.}\, J_0 \left(\frac{x}{\Lambda} \right) 
\end{equation}
kde $J_0$ je Besselova funkce prvního druhu. Difuzní délku lze vyjádřit jako

\begin{equation}
	\Lambda \approx \frac{r_0}{2,405}
\end{equation}
kde $r_0$ je poloměr výbojky a 2,405 je první kořen funkce $J_0$.
Objemovou rekombinaci můžeme napsat jako časovou změnu koncentrace, tedy
\begin{equation}
	\frac{\text{d}n}{\text{d}t} = -\alpha n^2
\end{equation}
kde $\alpha$ je koeficient rekombinace. Obecně platí, že rekombinační ztráty se více projevují při vysokém tlaku. Difuzní ztráty jsou naopak dominantní při nízkém tlaku a jsou charakterizovány časovou závislostí 

\begin{equation}
	n(t) = n_0\,\e^{\left( -\frac{Dt}{\Lambda^2}\right)}
\end{equation}
a tedy funkce $\ln n = f(t)$ je lineární, ze směrnice přímky lze určit $D$. V 
případě rekombinace platí

\begin{equation}
	\frac{1}{n(t)} = \frac{1}{n_0} + \alpha t
\end{equation} 
a závislost $1/\,n = f(t)$ je lineární, ze směrnice určíme $\alpha$. Pokud tímto způsobem určujeme jeden z koeficientů, tedy $D$ nebo $\alpha$, děláme to za předpokladu zanedbání druhého procesu. Nabízí se tedy vyjádřit $n(t)$ se zahrnutím obou koeficientů pomocí zpřesněné rovnice

\begin{equation}
	n(t) = \frac{1}{c\,\e^{\frac{tD}{\Lambda^2}}-\frac{\alpha \Lambda^2}{D}}
	\label{zpresnena}
\end{equation}

\subsection{Rezonátorová metoda stanovení koncentrace elektronů}
Pokud v rezonátoru zapálíme plazma, změní se jeho rezonanční frekvence $\omega$ i kvalita rezonátoru $Q$. Pro střední koncentraci elektronů $n$ ve výbojce o průměru $R'$ platí závislost na čase

\begin{equation}
	\overrightarrow{n}(t) = \frac{0,271 R^2 \Delta f(t) 8 \pi^2 \epsilon_0 m f_0}{0,64 R'^2 e^2}
	\label{koncentrace}
\end{equation}
kde $R$ je poloměr rezonátoru, $\Delta f(t)$ je rozdíl frekvencí zdroje $f'$~a~rezonanční frekvence prázdného rezonátoru 
$f_0$, $\epsilon_0$ je permitivita vakua, $m$ je hmotnost elektronu, $R'$ je 
poloměr výbojky a $e$ je elementární náboj.

\section{Měření a výsledky}
Měřící aparatura obsahuje vysokofrekvenční laditelný zdroj, který dodává energii do rezonátoru o poloměru $R = 40\,\si{\milli\meter}$, jehož osou prochází výbojka o poloměru $R' = 9\,\si{\milli\meter}$. Prošlý signál je na vstupu do osciloskopu usměrněný diodou. Proud měříme ampérmetrem, napětí osciloskopem. Výbojka je čerpána rotační olejovou a difuzní vývěvou, tlak měříme Piraniho manometrem. Ve výbojce máme helium jehož tlak lze měnit.

Rezonanční frekvence prázdného rezonátoru $f_0$ se po zapálení výboje zvýší na 
$f_1$. Po vypnutí přívodu energie se plazma začne rozpadat a rezonanční 
frekvence opět klesá až na původní hodnotu $f_0$. Tento periodický proces lze 
zachytit osciloskopem. Při měření měníme frekvenci zdroje $f'$ a z oscilogramu 
určujeme čas $t'$, za který dojde k rezonanci. Také si zaznamenáváme $f_0$, 
abychom následně mohli vypočítat koncentraci elektronů ze vztahu (\ref{koncentrace}). 
Následně můžeme graficky vynést závislosti $1/\,n = f(t)$ a $\ln n = f(t)$, 
určit z nich $\alpha$, $D$ a rozhodnout, zda je převládajícím procesem difúze 
nebo rekombinace. $\alpha$ a $D$ včetně $n_0$ také určíme proložením funkcí 
podle rovnice \eqref{zpresnena} a výsledky porovnáme.

Závislosti $1/\,n = f(t)$ a $\ln n = f(t)$ včetně proložení podle exponenciální rovnice 
\eqref{zpresnena} jsou vyneseny v grafech na obrázcích~\ref{g:5Pa}--\ref{g:450Pa}. 
Koeficienty $\alpha$ a $D$ určené z proložených lineárních a exponenciálních funkcí jsou 
uvedeny v tab.~\ref{table:koef}. Vidíme, že exponenciální fit podle rovnice \ref{zpresnena} je nejpřesnější, při našich podmínkách měření tedy probíhala rekombinace i difuze zároveň. Výsledné koeficienty se z lineárních fitů oproti exponenciálnímu fitu liší v některých případech i více než dvojnásobně. Pro tlaky v rozmezí 50--200 Pa jsou závislosti $\ln n = f(t)$ téměř lineární, převládá zde tedy difuze nad rekombinací. Pro vyšší tlak 450 Pa je lineárnější závislost $1/\,n = f(t)$, dominantní je rekombinace v objemu. Toto pozorování je v souladu s teorií. Pro tlaky 5--20 Pa není ani jedna ze závislostí lineární, nelze tak určit dominantní proces. 

\begin{figure}[h]
	\centering
	\begin{subfigure}[b]{.49\linewidth}
		\centering
		\scalebox{.30}{\includegraphics{graph1.png}}
		\caption{$f(t) = 1/n$}
	\end{subfigure}
	\begin{subfigure}[b]{.49\linewidth}
		\centering
		\scalebox{.30}{\includegraphics{graph2.png}}
		\caption{$f(t) = \ln n$}
	\end{subfigure}
	\caption{Časová závislost koncentrace elektronů pro tlak 5\,Pa.}
	\label{g:5Pa}
\end{figure}

\begin{figure}[h]
	\centering
	\begin{subfigure}[b]{.49\linewidth}
		\centering
		\scalebox{.30}{\includegraphics{graph3.png}}
		\caption{$f(t) = 1/n$}
	\end{subfigure}
	\begin{subfigure}[b]{.49\linewidth}
		\centering
		\scalebox{.30}{\includegraphics{graph4.png}}
		\caption{$f(t) = \ln n$}
	\end{subfigure}
	\caption{Časová závislost koncentrace elektronů pro tlak 10\,Pa.}
	\label{g:10Pa}
\end{figure}

\begin{figure}[h]
	\centering
	\begin{subfigure}[b]{.49\linewidth}
		\centering
		\scalebox{.30}{\includegraphics{graph5.png}}
		\caption{$f(t) = 1/n$}
	\end{subfigure}
	\begin{subfigure}[b]{.49\linewidth}
		\centering
		\scalebox{.30}{\includegraphics{graph6.png}}
		\caption{$f(t) = \ln n$}
	\end{subfigure}
	\caption{Časová závislost koncentrace elektronů pro tlak 20\,Pa.}
	\label{g:20Pa}
\end{figure}

\begin{figure}[h]
	\centering
	\begin{subfigure}[b]{.49\linewidth}
		\centering
		\scalebox{.30}{\includegraphics{graph7.png}}
		\caption{$f(t) = 1/n$}
	\end{subfigure}
	\begin{subfigure}[b]{.49\linewidth}
		\centering
		\scalebox{.30}{\includegraphics{graph8.png}}
		\caption{$f(t) = \ln n$}
	\end{subfigure}
	\caption{Časová závislost koncentrace elektronů pro tlak 50\,Pa.}
	\label{g:50Pa}
\end{figure}

\begin{figure}[h]
	\centering
	\begin{subfigure}[b]{.49\linewidth}
		\centering
		\scalebox{.30}{\includegraphics{graph9.png}}
		\caption{$f(t) = 1/n$}
	\end{subfigure}
	\begin{subfigure}[b]{.49\linewidth}
		\centering
		\scalebox{.30}{\includegraphics{graph10.png}}
		\caption{$f(t) = \ln n$}
	\end{subfigure}
	\caption{Časová závislost koncentrace elektronů pro tlak 100\,Pa.}
	\label{g:100Pa}
\end{figure}

\begin{figure}[h]
	\centering
	\begin{subfigure}[b]{.49\linewidth}
		\centering
		\scalebox{.30}{\includegraphics{graph11.png}}
		\caption{$f(t) = 1/n$}
	\end{subfigure}
	\begin{subfigure}[b]{.49\linewidth}
		\centering
		\scalebox{.30}{\includegraphics{graph12.png}}
		\caption{$f(t) = \ln n$}
	\end{subfigure}
	\caption{Časová závislost koncentrace elektronů pro tlak 200\,Pa.}
	\label{g:200Pa}
\end{figure}

\begin{figure}[h]
	\centering
	\begin{subfigure}[b]{.49\linewidth}
		\centering
		\scalebox{.30}{\includegraphics{graph13.png}}
		\caption{$f(t) = 1/n$}
	\end{subfigure}
	\begin{subfigure}[b]{.49\linewidth}
		\centering
		\scalebox{.30}{\includegraphics{graph14.png}}
		\caption{$f(t) = \ln n$}
	\end{subfigure}
	\caption{Časová závislost koncentrace elektronů pro tlak 450\,Pa.}
	\label{g:450Pa}
\end{figure}



\begin{table}[h]
	\centering
	\caption{Hodnoty koeficientu rekombinace $\alpha$ a difuzního koeficientu 
	$D$ určené ze závislostí $1/n = f(t)$, $\ln n = f(t)$ a kombinované dle 
	rovnice \eqref{zpresnena}.}
	\label{table:koef}
	\begin{tabular}{|r|c|c|c|c|}
		\hline
		Tlak    & $1/n = f(t)$      & $\ln n = f(t)$ & 
		\multicolumn{2}{c|}{Obě rekombinace}                                \\ 
		\hline
		{[}Pa{]} & 
		$\alpha$\,[\si{\per\second}]$\cdot10^{-3}$ & 
		$D$\,[$\si{\metre^2\second^{-1}}$]$\cdot10^{-3}$              & 
		$\alpha$\,[$\si{\second^{-1}}$]$\cdot10^{-3}$ & 
		$D$\,[$\si{\metre^2\second^{-1}}$]$\cdot10^{-3}$ \\ 
		\hline
		$5$                              &$ 2.01                 \pm 
		0.11                $&$ 57.3              \pm  2.8              $& 
		$0.7734    
		\pm 0.0004    $&$ 28.4       \pm  0.4       $\\ \hline
		$10$                             & $1.21                 \pm  
		0.05                $&$ 50.7              \pm  3.3              $&$ 
		0.576     
		\pm 0.006     $&$ 21.8       \pm  1.5       $\\ \hline
		$20                             $&$ 1.00                 \pm  
		0.06                $&$ 56.9              \pm  2.7              $&$ 
		0.436     
		\pm 0.003     $&$ 25.2       \pm  1.3       $\\ \hline
		$50                             $&$ 0.82                 \pm  
		0.07                $&$ 52.7              \pm  1.0              $&$ 
		0.180     
		\pm 0.003     $&$ 37.1       \pm  2.2       $\\ \hline
		$100                            $&$ 0.30                 \pm  
		0.02                $&$ 36.1              \pm  0.5              $& 
		$0.0292    
		\pm 0.0002    $&$ 31.6       \pm  0.8       $\\ \hline
		$200                            $&$ 0.27                 \pm  
		0.02                $&$ 23.2              \pm  0.6              $& 
		$0.0590    
		\pm 0.0004    $&$ 16.8       \pm  0.7       $\\ \hline
		$450                            $&$ 0.21                 \pm  
		0.01                $&$ 15.5              \pm  0.8              $& 
		$0.118     
		\pm 0.003     $&$ 5.6        \pm  0.7       $\\ \hline
	\end{tabular}
\end{table}
\clearpage
\section{Závěr}
V této úloze jsme se zabývali rozpadem plazmatu a popsali jsme procesy, jakým k němu dochází. Naším úkolem bylo určit koncentraci elektronů v závislosti na čase, za který dojde k rezonanci. Z těchto závislostí, které jsme naměřili pro tlaky v rozmezí 5--450 Pa, jsme fitováním třemi různými funkcemi určili koeficienty rekombinace a difuzní koeficienty. Naše výsledky se pro oblast 50--450 Pa shodují s teorií. Při nízkém tlaku do 200 Pa je dominantním procesem difuze, při 450 Pa je to naopak rekombinace. Pro tlaky 5--20 Pa jsme z našich dat převládající proces určit nedokázali.


\end{document}
