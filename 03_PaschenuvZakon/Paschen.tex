\documentclass[a4paper,12pt]{article}
\usepackage [utf8x]{inputenc}
\usepackage[czech]{babel}
\usepackage{graphicx}
\usepackage{amsmath}
\usepackage{siunitx}
\usepackage{xspace}
\usepackage{url}
\usepackage{indentfirst}
\usepackage[margin=22mm]{geometry}
\usepackage{esvect}
\usepackage{ragged2e}
\usepackage{tikz,pgf}
\usepackage{bm}
\usepackage{perpage}
\usepackage{capt-of}

\graphicspath{
	{img/}
	{plots/}
}

\newcommand{\e}{\text{e}}


\MakeSorted{figure}
\newtoks\jmenopraktika \newtoks\jmeno \newtoks\datum
\newtoks\obor \newtoks\skupina \newtoks\rocnik \newtoks\semestr
\newtoks\cisloulohy \newtoks\jmenoulohy
\newtoks\tlak \newtoks\teplota \newtoks\vlhkost
\jmenopraktika={Paschenův zákon, katodový spád potenciálu v plazmatu}  % nahradte jmenem vaseho predmetu
\jmeno={Radek Horňák, Lukáš Vrána}            
\datum={15. 3. 2022}        % nahradte datem mereni ulohy                           
\rocnik={2.}                  
\semestr={IV.}                 
\cisloulohy={6}    % cislo ulohy           

\begin{document}
	\begin{center}
		{\Large Přírodovědecká fakulta Masarykovy univerzity} \\
		\bigskip
		{\Large \bfseries PRAKTIKUM Z~FYZIKY PLAZMATU} \\
		\bigskip
		{\Large \the\jmenopraktika}
	\end{center}
	\bigskip
	\noindent
	\setlength{\arrayrulewidth}{1pt}
	\begin{tabular*}{\textwidth}{@{\extracolsep{\fill}} l l}
		\large {\bfseries Zpracovali:}  \the\jmeno  \hspace{20mm} \large  
		{\bfseries Naměřeno:} \the\datum\\[2.5mm]
		\hline
	\end{tabular*}

\section{Teorie}


\section{Měření a výsledky}


\section{Závěr}


\end{document}
