\documentclass[a4paper,12pt]{article}
\usepackage [utf8x]{inputenc}
\usepackage[czech]{babel}
\usepackage{graphicx}
\usepackage{amsmath}
\usepackage{siunitx}
\usepackage{xspace}
\usepackage{url}
\usepackage{indentfirst}
\usepackage[margin=22mm]{geometry}
\usepackage{esvect}
\usepackage{ragged2e}
\usepackage{tikz,pgf}
\usepackage{bm}
\usepackage{perpage}
\usepackage{capt-of}

\graphicspath{
	{img/}
	{plots/}
}

\newcommand{\e}{\text{e}}


\MakeSorted{figure}
\newtoks\jmenopraktika \newtoks\jmeno \newtoks\datum
\newtoks\obor \newtoks\skupina \newtoks\rocnik \newtoks\semestr
\newtoks\cisloulohy \newtoks\jmenoulohy
\newtoks\tlak \newtoks\teplota \newtoks\vlhkost
\jmenopraktika={Paschenův zákon, katodový spád potenciálu v doutnavém výboji}  % nahradte jmenem vaseho predmetu
\jmeno={Radek Horňák, Lukáš Vrána}            
\datum={15. 3. 2022}        % nahradte datem mereni ulohy                           
\rocnik={2.}                  
\semestr={IV.}                 
\cisloulohy={6}    % cislo ulohy           

\begin{document}
	\begin{center}
		{\Large Přírodovědecká fakulta Masarykovy univerzity} \\
		\bigskip
		{\Large \bfseries PRAKTIKUM Z~FYZIKY PLAZMATU} \\
		\bigskip
		{\Large \the\jmenopraktika}
	\end{center}
	\bigskip
	\noindent
	\setlength{\arrayrulewidth}{1pt}
	\begin{tabular*}{\textwidth}{@{\extracolsep{\fill}} l l}
		\large {\bfseries Zpracovali:}  \the\jmeno  \hspace{20mm} \large  
		{\bfseries Naměřeno:} \the\datum\\[2.5mm]
		\hline
	\end{tabular*}

\section{Teorie}
\subsection{Paschenův zákon}

Z Townsendovy teorie lavin víme, že působením elektrického pole na zředěný plyn dochází k urychlování přítomných elektronů. Takto urychlené elektrony mohou ionizovat neutrální částice a vytvořit takzvanou Townsedovu lavinu. Počet elektronů vzniklých v důsledku Townsendovy laviny závisí exponenciálně na dráze $d$  

\begin{equation}
	n = n_0\,\e^{\alpha d}
	\label{1}
\end{equation}

kde $n_0$ je počet elektronů v~počátečním bodě $d = 0$ a $\alpha$ je první Townsendův nebo také ionizační koeficient. Elektrické pole můžeme charakterizovat napětím $V$ přiloženým mezi dvě rovinné elektrody, dráha $d$ je vzdálenost mezi elektrodami. Elektronovou lavinu doprovází vznik iontů, jejichž počet lze vyjádřit jako

\begin{equation}
	 n_i = n_0\,[e^{\alpha d}-1]
	\label{2}
\end{equation}

Ionty jsou polem urychlovány ke katodě, dopadají na ni a vyvolávají sekundární emisi elektronů. Tu popisuje Townsendův třetí koeficient neboli koeficient sekundární emise $\gamma$. Konkrétně udává průměrný počet elektronů emitovaných jedním iontem při jeho dopadu na katodu. Pomocí $\gamma$ lze vyjádřit podmínku zapálení výboje jako

\begin{equation}
	\gamma\,(e^{\alpha d } - 1) = 1
	\label{3}
\end{equation} 

tedy že v lavině musí být jedním primárním elektronem vytvořeno tolik iontů, které dopadem na katodu způsobí emisi jednoho nového elektronu. Ionizační koeficient $\alpha$ závisí na intenzitě elektrického pole 

\begin{equation}
	\frac{\alpha}{p} = A\,\e^{-\frac{Bp}{E}} 
	\label{4}
\end{equation}

kde $A = 1/\lambda_1$ a $B = U_i/\lambda_1$ jsou konstanty závislé na druhu plynu, $\lambda_1$ je střední volná dráha elektronů při jednotkovém tlaku. Dále lze (\ref{4}) přepsat jako

\begin{equation}
	\frac{\alpha}{p} = A\,\e^{-\frac{Bpd}{V}} 
	\label{5}
\end{equation}

Logaritmováním a úpravou (\ref{5}) získáme

\begin{equation}
	V = \frac{B\,pd}{\ln A - \ln \frac{\alpha}{d}}
	\label{6}
\end{equation}

Dosazením $\alpha\,d$ z (\ref{6}) do (\ref{3}), logaritmováním a dalšími úpravami dojdeme k tvaru

\begin{equation}
	A\,pd\,e^{-\frac{Bpd}{V_z}} = \ln \left(\frac{1}{\gamma} + 1\right)
	\label{7}
\end{equation}

kde $V_z$ je zápalné napětí výboje. Pro daný plyn a materiál katody položme pravou stranu (\ref{7})

\begin{equation}
	\ln \left(\frac{1}{\gamma} + 1\right) = C
	\label{8}
\end{equation}

Úpravami dostáváme

\begin{equation}
	V_z = \frac{B\,pd}{C' + \ln(pd)}
	\label{9}
\end{equation}

kde $C' = \ln C - \ln A$. Závislost $V_z = f (pd))$ se nazývá Paschenův zákon. Má charakteristický tvar, včetně minima nazývaného Stoletowův bod.

%odsazení zkontroluj

\subsection{Katodový spád potenciálu v doutnavém výboji}

\section{Měření a výsledky}



\section{Závěr}


\end{document}
