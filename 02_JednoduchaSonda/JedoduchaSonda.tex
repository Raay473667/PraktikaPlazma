\documentclass[a4paper,12pt]{article}
\usepackage [utf8x]{inputenc}
\usepackage[czech]{babel}
\usepackage{graphicx}
\usepackage{amsmath}
\usepackage{siunitx}
\usepackage{xspace}
\usepackage{url}
\usepackage{indentfirst}
\usepackage[margin=22mm]{geometry}
\usepackage{esvect}
\usepackage{ragged2e}
\usepackage{tikz,pgf}
\usepackage{bm}
\usepackage{perpage}
\usepackage{capt-of}
\usepackage{subcaption}

\graphicspath{
	{img/}
	{plots/}
}

\newcommand{\e}{\text{e}}


\MakeSorted{figure}
\newtoks\jmenopraktika \newtoks\jmeno \newtoks\datum
\newtoks\obor \newtoks\skupina \newtoks\rocnik \newtoks\semestr
\newtoks\cisloulohy \newtoks\jmenoulohy
\newtoks\tlak \newtoks\teplota \newtoks\vlhkost
\jmenopraktika={Diagnostika plazmatu doutnavého výboje pomocí jednoduché sondy}  % nahradte jmenem vaseho predmetu
\jmeno={Radek Horňák, Lukáš Vrána}            
\datum={1. 3. 2022}        % nahradte datem mereni ulohy                           
\rocnik={2.}                  
\semestr={IV.}                 
\cisloulohy={6}    % cislo ulohy           

\begin{document}
	\begin{center}
		{\Large Přírodovědecká fakulta Masarykovy univerzity} \\
		\bigskip
		{\Large \bfseries PRAKTIKUM Z~FYZIKY PLAZMATU} \\
		\bigskip
		{\Large \the\jmenopraktika}
	\end{center}
	\bigskip
	\noindent
	\setlength{\arrayrulewidth}{1pt}
	\begin{tabular*}{\textwidth}{@{\extracolsep{\fill}} l l}
		\large {\bfseries Zpracovali:}  \the\jmeno  \hspace{20mm} \large  
		{\bfseries Naměřeno:} \the\datum\\[2.5mm]
		\hline
	\end{tabular*}

\section{Teorie}
%\subsection{Doutnavý výboj}
%V této úloze se budeme zabývat měřením parametrů kladného sloupce doutnavého výboje.	
%Doutnavý výboj je druh výboje, jehož typický vzhled můžeme vidět na schématu na 
%obr.~\ref{glow+napeti}. V~rozmezí tlaku 10$^1$-10$^2$\,\si{\pascal} v~něm můžeme 
%pozorovat střídající se temné nebo svítící místa, která rozlišujeme na katodové a anodové oblasti. Katodové oblasti jsou nutnou a důležitou oblastí doutnavého výboje na rozdíl od kladného sloupce, jehož délka se při zkracování vzdálenosti elektrod zmenšuje a může i úplně zmizet. Intenzita elektrického pole je podél jeho osy konstantní, viz dolní část obr. \ref{glow+napeti}. Může být určena buď pomocí sondových měření nebo ji lze vypočítat ze závislosti napětí na elektrodách na jejich vzdálenosti při konstantním proudu ve výboji.

%\begin{figure}[h]
%	\centering
%	\includegraphics[width=110mm]{glow+napeti.png}
%	\caption{Doutnavý výboj a intenzita elektrického pole podíl jeho osy}
%	\label{glow+napeti}
%\end{figure}

\subsection{Elektrostatická Langmuirova sonda}
Langmuirova sonda je vodič malých rozměrů zavedený do plazmatu, pomocí nějž lze měřit
nejdůležitější parametry plazmatu jako elektronovou hustotu $n_\text{e}$, teplotu elektronů
$T_\text{e}$, rozdělovací funkci elektronů $f(v)$ a prostorové rozdělení potenciálu a
elektrického pole. Napětí sondy $U_\text{S}$ určujeme vzhledem k referenční elektrodě. Potenciál plazmatu v místě sondy vůči stejné referenční elektrodě označme $V_\text{p}$. Pokud je vůči ní plocha sondy velmi malá, můžeme sondu nazvat jednoduchou. Podle tvaru lze dále sondy dělit na válcové, kulové a~rovinné. Závislost proudu protékajícího sondou $I_\text{S}$ na napětí přiloženém na sondu $V_\text{S}$ tvoří voltampérovou (VA) charakteristiku sondy. Napětí sondy $U_\text{S}$ získáme pomocí vztahu

\begin{equation}
	U_\text{S} = V_\text{S} - V_\text{p}
	\label{Usondy}
\end{equation}
Pokud sonda není připojena k vnějšímu obvodu a proud elektronů i iontů na ni se ustálí, je výsledný proud nulový a sonda se ustálí na napětí $V_\text{{fl}}$, tedy na plovoucím potenciálu.

VA charakteristiku jednoduché sondy můžeme rozdělit na tři části. Tou první je oblast saturovaného iontového proudu označená na obr.
\ref{VA} jako $A$. Sonda je záporně nabita vzhledem k potenciálu plazmatu, elektrony jsou odpuzovány a ionty naopak přitahovány. Vizuálně se to projevuje temným prostorem obalujícím sondu.

Druhou část charakteristiky tvoří přechodová oblast, pro kterou lze $U_\text{S}$ vymezit jako $-2(V_\text{p} - V_\text{{fl}} \leq
U_\text{S} \leq 0)$. Na obr. \ref{VA} se jedná o oblast $B$. Celkový proud sondou $I_S$ můžeme vyjádřit jako

\begin{equation}
	I_\text{S} = I_\text{i} + I_\text{e}
\end{equation}
kde $I_\text{i}$ je iontový proud a $I_\text{e}$ elektronový proud, který je dán vztahem

\begin{equation}
	I_\text{e} = S e n_\text{e} \sqrt{\frac{k T_\text{e}}{2 \pi m_\text{e}}} \exp \left(\frac{-eU_\text{S}}{k T_\text{e}}\right)
	\label{eproud}
\end{equation}
kde $S$ je povrch sondy, $e$ elementární náboj, $n_\text{e}$ koncentrace elektronů, $k$ Boltzmanova konstanta a $m_\text{e}$ hmotnost
elektronu.

Oblast saturovaného elektronového proudu je na obr. \ref{VA} označená jako $C$. Sonda je vzhledem k potenciálu plazmatu na kladném
napětí a přitahuje tak elektrony. U válcové sondy nejeví tato oblast nasycení, nýbrž parabolicky narůstá.


\begin{figure}[h]
	\centering
	\includegraphics[width=90mm]{VA.png}
	\caption{VA charakteristika jednoduché rovinné sondy.}
	\label{VA}
\end{figure}	
\newpage
\section{Měření a výsledky}
Měření provádíme na aparatuře, jejíž schéma je vidět na obr. \ref{schema}. Výbojka je čerpaná rotační olejovou vývěvou. Tlak
nastavujeme změnou průtoku argonu a měříme jej Piraniho manometrem. Do výbojky je zavedená jednoduchá válcová sonda, jejíž délku jsme
odhadli na 8 \si{\milli\meter} a průměr 0,1 \si{\milli\meter}. Povrch podstavy válcové sondy je k povrchu jejího pláště $S$~zanedbatelný, po zaokrouhlení dostáváme $S = 2,5\cdot10^{-6}$ \si{\meter\squared}. 

Při měření vždy nejprve nalezneme plovoucí potenciál, abychom měli jistotu, že naměříme oblast nalevo i napravo od něj. Napětí přiložené na sondu $V_\text{S}$ se mění automaticky pomocí elektrického motorku, na kterém stačí zařadit rychlostní stupeň v jednom
se směrů chodu. Data jsou ukládána na počítač. Při vyhodnocování jsme je museli synchronizovat.   

\begin{figure}[h]
	\centering
	\includegraphics[width=120mm]{schema.png}
	\caption{Schéma aparatury.}
	\label{schema}
\end{figure}

Provedli jsme měření za konstantního tlaku 100 \si{\pascal} pro tři hodnoty výbojového proudu $I_\text{v}$. Výsledné VA charakteristiky
jsou v grafu na obr. \ref{namerene012}. Z nich lze určit plovoucí potenciál, který se s rostoucím výbojovým proudem zvětšuje, viz tab.
\ref{tab1}. Dále jsme provedli měření za konstantního výbojového proudu 40 \si{\milli\ampere} pro pět hodnot tlaku. Odpovídající VA
charakteristiky jsou v grafu na obr. \ref{namerene34567}. Pro tlak 200 \si{\pascal} je plovoucí potenciál nejmenší, v oblasti 5--50
\si{\pascal} však nevykazuje žádný trend, viz tab. \ref{tab1}.

\begin{figure}[h!]
	\centering
	\includegraphics[width=145mm]{namerene012.png}
	\caption{Naměřené VA charakteristiky za konstantního tlaku 100 \si{\pascal}.}
	\label{namerene012}	
\end{figure}

\begin{figure}[h!]
	\centering
	\includegraphics[width=145mm]{namerene34567.png}
	\caption{Naměřené VA charakteristiky za konstantního výbojového proudu 40 \si{\milli\ampere}.}
	\label{namerene34567}
\end{figure}

\newpage
Nyní je potřeba od charakteristik odečíst iontový proud, oblast kde saturuje jsme proložili křivkou. Názorné proložení pro VA
charakteristku za podmínek $p = 100$ \si{\pascal} a $I_\text{v}$ = 40 \si{\milli\ampere} je na obr. \ref{iiont}. Ve zbylých případech
jsme postupovali obdobně. VA charakteristiky s takto odečteným iontovým proudem jsou v grafech na obr. \ref{odectene012} a
\ref{odectene34567}.

\begin{figure}[h]
	\centering
	\includegraphics[width=145mm]{iiont.png}
	\caption{Lineární fit saturovaného iontového proudu, $p = 100$ \si{\pascal} a $I_\text{v} = 40$ \si{\milli\ampere}.}
	\label{iiont}
\end{figure}

\newpage
\begin{figure}[h!]
	\centering
	\includegraphics[width=135mm]{odectene012.png}
	\caption{VA charakteristiky s odečteným iontovým proudem pro měření s konstantním tlakem $p = 100$ \si{\pascal}.}
	\label{odectene012}
\end{figure}

\begin{figure}[h!]
	\centering
	\includegraphics[width=135mm]{odectene34567.png}
	\caption{VA charakteristiky s odečteným iontovým proudem pro měření s konstantním proudem $I_\text{v} = 40$ \si{\milli\ampere}.}
	\label{odectene34567}
\end{figure}

\newpage
Potenciál plazmatu $V_\text{p}$ přibližně určíme ze zlomu VA charakteristik jako průsečík asymptot k lineárním částem závislostí. Tento
postup je vidět na obrázcích \ref{data1} až \ref{data7} vlevo a výsledné $V_\text{p}$~jsou uvedeny v tab. \ref{tab1}. Vždy platí, že
$V_\text{p}$ je větší než $V_\text{fl}$. Stejně jako $V_\text{fl}$, potenciál plazmatu s~rostoucím výbojovým proudem roste, při změně
tlaku nevykazuje žádný trend. Nyní můžeme ze vztahu \eqref{Usondy} dopočítat $U_\text{S}$. Pokud následně vyneseme do grafů závislosti $\ln
I_{\text{e}}=-\frac{e}{kT_e}U_\text{S}+C$~pro oblasti $-2(V_\text{p} - V_\text{{fl}} \leq U_\text{S} \leq 0)$, můžeme z
elektronového proudu pro $U_\text{S} = 0$ dle vztahu \eqref{eproud} dopočítat koncentraci elektronů. Závislosti $\ln I_{\text{e}} =
f(U_\text{S})$ proložené přímkou jsou na obrázcích \ref{data1} až \ref{data7} vpravo. Výsledné elektronové teploty a koncentrace
elektronů jsou v tab. \ref{tab2}. S rostoucím výbojovým proudem roste i koncentrace elektronů a jejich teplota klesá. S rostoucím
tlakem pozorujeme stejnou závislost, tedy klesající teplotu a rostoucí koncentrace elektronů.

\begin{center}
	\begin{table}[h!]
		\centering
		\caption{Plovoucí a plazmové potenciály}
		\label{tab1}
		\begin{tabular}{|c|c|c|c|c|c|} \hline
			\multicolumn{1}{|c|}{}  & \multicolumn{2}{c|}{$p$ = 100 \si{\pascal}}& \multicolumn{1}{|c|}{} & \multicolumn{2}{c|}{$I_\text{v}$ = 40 \si{\milli\ampere} }  \\ \hline
			$I_\text{v}$ [\si{\milli\ampere}] &  $V_\text{fl}$ [V] & $V_\text{p}$ [V] & $p$ [\si{\pascal}] &  $V_\text{fl}$ [V] & $V_\text{p}$ [V] \\ \hline
			30 & -48,0 & -47,7 & 5 & -45,3 & -44,8\\ \hline
			40 & -43,8 & -43,4 & 10 & -45,8 & -45,2 \\ \hline
			50 & -42,2 & -41,6 & 20 & -45,0 & -44,6 \\ \hline
			&  &  & 50 & -44,4 & -43,9 \\ \hline
			& &  & 200 & -50,9 & -49,9 \\ \hline
			
		\end{tabular}
	\end{table}
\end{center}


\begin{center}
	\begin{table}[h!]
		\centering
		\caption{Teploty a koncentrace elektronů}
		\label{tab2}
		\begin{tabular}{|c|c|c|c|c|c|} \hline
			\multicolumn{1}{|c|}{}  & \multicolumn{2}{c|}{$p$ = 100 \si{\pascal}}& \multicolumn{1}{|c|}{} & \multicolumn{2}{c|}{$I_\text{v}$ = 40 \si{\milli\ampere} }  \\ \hline
			$I_\text{v}$ [\si{\milli\ampere}] &  $T$ [eV] & $n_\text{e} [10^{14}\si{\per\meter}$] & $p$ [\si{\pascal}] &  $T$ [eV] & $n_\text{e} [10^{14}\si{\per\meter}]$ \\ \hline
			30 & 3,3 & 1,0 & 5 & 4,6 & 0,8\\ \hline
			40 & 2,8 & 1,6 & 10 & 4,3 & 1,1 \\ \hline
			50 & 2,6 & 2,3 & 20 & 4,0 & 1,2\\ \hline
			&  &  & 50 & 3,7 & 1,4 \\ \hline
			& &  & 200 & 2,2 & 1,7 \\ \hline
			
		\end{tabular}
	\end{table}
\end{center}
\newpage
\begin{figure}[h]
	\centering
	\begin{subfigure}[b]{.49\textwidth}
		\centering
		\scalebox{.34}{\includegraphics{data1Vp.png}}
		%\caption{$f(t) = 1/n$}
	\end{subfigure}
	\begin{subfigure}[b]{.49\textwidth}
		\centering
		\scalebox{.34}{\includegraphics{data1T.png}}
		%\caption{$f(t) = \ln n$}
	\end{subfigure}
	\caption{Stanovení potenciálu plazmatu a elektronové teploty, $p = 100$ \si{\pascal} a $I_\text{v} = 30$ \si{\milli\ampere}.}
	\label{data1}
\end{figure}

\begin{figure}[h]
	\centering
	\begin{subfigure}[b]{.49\textwidth}
		\centering
		\scalebox{.34}{\includegraphics{data0Vp.png}}
		%\caption{$f(t) = 1/n$}
	\end{subfigure}
	\begin{subfigure}[b]{.49\textwidth}
		\centering
		\scalebox{.34}{\includegraphics{data0T.png}}
		%\caption{$f(t) = \ln n$}
	\end{subfigure}
	\caption{Stanovení potenciálu plazmatu a elektronové teploty, $p = 100$ \si{\pascal} a $I_\text{v} = 40$ \si{\milli\ampere}.}
	\label{data0}
\end{figure}

\newpage
\begin{figure}[h]
	\centering
	\begin{subfigure}[b]{.49\textwidth}
		\centering
		\scalebox{.34}{\includegraphics{data2Vp.png}}
		%\caption{$f(t) = 1/n$}
	\end{subfigure}
	\begin{subfigure}[b]{.49\textwidth}
		\centering
		\scalebox{.34}{\includegraphics{data2T.png}}
		%\caption{$f(t) = \ln n$}
	\end{subfigure}
	\caption{Stanovení potenciálu plazmatu a elektronové teploty, $p = 100$ \si{\pascal} a $I_\text{v} = 50$ \si{\milli\ampere}.}
	\label{data2}
\end{figure}



\begin{figure}[h]
	\centering
	\begin{subfigure}[b]{.49\textwidth}
		\centering
		\scalebox{.34}{\includegraphics{data3Vp.png}}
		%\caption{$f(t) = 1/n$}
	\end{subfigure}
	\begin{subfigure}[b]{.49\textwidth}
		\centering
		\scalebox{.34}{\includegraphics{data3T.png}}
		%\caption{$f(t) = \ln n$}
	\end{subfigure}
	\caption{Stanovení potenciálu plazmatu a elektronové teploty, $p = 200$ \si{\pascal} a $I_\text{v} = 40$ \si{\milli\ampere}.}
	\label{data3}
\end{figure}

\newpage
\begin{figure}[h]
	\centering
	\begin{subfigure}[b]{.49\textwidth}
		\centering
		\scalebox{.34}{\includegraphics{data4Vp.png}}
		%\caption{$f(t) = 1/n$}
	\end{subfigure}
	\begin{subfigure}[b]{.49\textwidth}
		\centering
		\scalebox{.34}{\includegraphics{data4T.png}}
		%\caption{$f(t) = \ln n$}
	\end{subfigure}
	\caption{Stanovení potenciálu plazmatu a elektronové teploty, $p = 50$ \si{\pascal} a $I_\text{v} = 40$ \si{\milli\ampere}.}
	\label{data4}
\end{figure}

\begin{figure}[h]
	\centering
	\begin{subfigure}[b]{.49\textwidth}
		\centering
		\scalebox{.34}{\includegraphics{data5Vp.png}}
		%\caption{$f(t) = 1/n$}
	\end{subfigure}
	\begin{subfigure}[b]{.49\textwidth}
		\centering
		\scalebox{.34}{\includegraphics{data5T.png}}
		%\caption{$f(t) = \ln n$}
	\end{subfigure}
	\caption{Stanovení potenciálu plazmatu a elektronové teploty, $p = 20$ \si{\pascal} a $I_\text{v} = 40$ \si{\milli\ampere}.}
	\label{data5}
\end{figure}

\newpage
\begin{figure}[h!]
	\centering
	\begin{subfigure}[b]{.49\textwidth}
		\centering
		\scalebox{.34}{\includegraphics{data6Vp.png}}
		%\caption{$f(t) = 1/n$}
	\end{subfigure}
	\begin{subfigure}[b]{.49\textwidth}
		\centering
		\scalebox{.34}{\includegraphics{data6T.png}}
		%\caption{$f(t) = \ln n$}
	\end{subfigure}
	\caption{Stanovení potenciálu plazmatu a elektronové teploty, $p = 10$ \si{\pascal} a $I_\text{v} = 40$ \si{\milli\ampere}.}
	\label{data6}
\end{figure}

\begin{figure}[h!]
	\centering
	\begin{subfigure}[b]{.49\textwidth}
		\centering
		\scalebox{.34}{\includegraphics{data7Vp.png}}
		%\caption{$f(t) = 1/n$}
	\end{subfigure}
	\begin{subfigure}[b]{.49\textwidth}
		\centering
		\scalebox{.34}{\includegraphics{data7T.png}}
		%\caption{$f(t) = \ln n$}
	\end{subfigure}
	\caption{Stanovení potenciálu plazmatu a elektronové teploty, $p = 5$ \si{\pascal} a $I_\text{v} = 40$ \si{\milli\ampere}.}
	\label{data7}
\end{figure}

\section{Závěr}
V této úloze jsme se seznámili s měřením pomocí Langmuirovy jednoduché válcové sondy. Naměřili jsme osm VA charakteristik pro různé
podmínky. Určili jsme plovoucí potenciál sondy, který se zvětšuje s rostoucím výbojovým proudem, při změnách tlaku za konstantního
proudu nevykazoval žádný trend. Dále jsme určili potenciál plazmatu, ten je vždy větší než plovoucí potenciál a při změnách výbojového
proudu a tlaku se chová obdobně jako plovoucí potenciál. Nakonec jsme získali elektronové teploty a spočítali elektronovou koncentraci. S rostoucím výbojovým
proudem roste i koncentrace elektronů a jejich teplota klesá. S rostoucím tlakem jsme pozorovali stejnou závislost, tedy rostoucí
koncentraci elektronů a~klesající elektronovou teplotu.
\end{document}
